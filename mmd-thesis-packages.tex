%
%   Default packages for thesis documents created by MultiMarkdown
%

\usepackage{microtype}              % Typography optimization on micro-level
\usepackage{fancyvrb}			    % Allow \verbatim et al. in footnotes
\usepackage{amsfonts}
\usepackage{amsmath}
% \usepackage[nosectionbib]{apacite}
\usepackage{siunitx}

\usepackage{booktabs}               % Better tables
\usepackage{tabulary}               % Support longer table cells
\usepackage{multirow}
\usepackage[bb=pazo]{mathalfa}      % Better support for blackboard bold font etc
\usepackage{centernot}
\usepackage{fnpos}                  % \makeFNbelow by default

% \usepackage[parfill]{parskip}       % Use line breaks instead of indentation

\usepackage[sort&compress]{natbib}  % Better bibliography support

\usepackage{acronym}
\usepackage[left=3.5cm, right=2cm, bottom=3.5cm]{geometry}
\geometry{a4paper}

% \usepackage{helvet}                 % Use Helvetica (for sans-serif)
% \usepackage{mathpazo}               % Use Palatino (for serif)
\usepackage{textcomp}               % Euro-Symbol
\usepackage{appendix}

\usepackage[T1]{fontenc}		    % Use T1 font encoding for accented characters
\usepackage[utf8x]{inputenc}        % For UTF-8 support

\usepackage{boxedminipage}          % Surround parts of graphics with box
\usepackage[labelfont=bf]{caption}
\usepackage{subcaption}             % Use sub-figures

\usepackage{xcolor}				    % Allow for color (annotations)
\usepackage{listings}			    % Allow for source code highlighting

% This is now the recommended way for checking for PDFLaTeX:
\usepackage{ifpdf}

\ifpdf%
\usepackage[pdftex]{graphicx}
\else
\usepackage{graphicx}
\fi

% Packages that include graphicx
\usepackage{todonotes}              % Use TODO notes
\usepackage{xfrac}
\usepackage{pgfplots}               % Support TikZ graphics
